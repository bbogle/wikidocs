% !TEX TS-program = pdflatex

\documentclass{book}
\usepackage[paperwidth=165.8mm, paperheight=231mm, hmarginratio=1:1]{geometry}

\usepackage[hangul,nonfrench,finemath]{kotex}

\usepackage{booktabs}
\usepackage{fancyhdr}
\usepackage{tocloft}

% hyperref 설정
\usepackage[unicode, colorlinks, breaklinks, bookmarks, pdftex]{hyperref}
\hypersetup {
  pdftitle={$title$},
  pdfauthor={$author$},
  linkcolor=black
}

% 한글 설정들
\usepackage[default]{dhucs-interword}
\usehangulfontspec{ut}
\usepackage[hangul]{dhucs-setspace}
\usepackage{dhucs-gremph}

% ??
\usepackage{amsmath,amssymb}

% 이미지 삽입을 위해 필요
\usepackage{graphicx}

% 테이블 기능
\usepackage{longtable}

% 단락시작시 들여쓰기 방지
\usepackage{parskip}

% 목차 번호 방지
\setcounter{secnumdepth}{0}
\let\stdsection\section
\renewcommand\section{\newpage\stdsection}

% 한글폰트 변경하기?
%\SetHangulFonts{Daum_Regular}{NanumGothic}{NanumGothicCoding}


% PDF 제목, 저자 추가하기
\usepackage{etoolbox}
\makeatletter
\AtBeginDocument{
  \hypersetup{
    pdftitle = {\@title},
    pdfauthor = {\@author}
  }
}
\makeatother
\title{BOOK_SUBJECT}
\author{BOOK_AUTHOR}


% 목차는 2depth만 표시되도록
\setcounter{tocdepth}{2}

\tocloftpagestyle{empty}

\begin{document}

  \pagestyle{empty}

  %\vspace*{\fill}
  %  \begin{center}
  %    \includegraphics[width=0.7\textwidth]{TITLE}
  %  \end{center}
  %\vspace*{\fill}

  % 제목과 저자 그리고 날짜
  \maketitle

  % 목차
  \tableofcontents

  % 페이징
  \pagestyle{fancy}
  % Delete the current section for header and footer
  \fancyhf{}
  % Set custom header
  \lhead[]{\thepage}
  \rhead[\thepage]{}
  % Set arabic (1,2,3...) page numbering
  \pagenumbering{arabic}

  % Book contents
  $body$

\end{document}
